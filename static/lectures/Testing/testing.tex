\documentclass[handout]{beamer}
%\documentclass[]{beamer}
%\usepackage[dvips]{color}
\usepackage{graphicx}
\usepackage{amsmath,amssymb,array,comment,eucal}
\input{../macros}
\usepackage{verbatim}


\title{Hypothesis Testing and Model Choice}
\subtitle{Merlise Clyde}
\author{STA721 Linear Models}
\institute{Duke University}
\date{\today}
\logo{duke.eps}

\begin{document}
\maketitle

\begin{frame}\frametitle{Outline}
Topics
  \begin{itemize}
  \item Climate Example
  \item t-tests
  \item Overall F-test
  \item Sequential F-tests
  \item Added Variable Plots (if time)
  \item Summary

  \end{itemize}


Readings: Christensen Chapter 2 (section 7),  Chapter 10, Appendix B
\end{frame}

\begin{frame}
  \frametitle{Climate Change ?}
  Scientists are interested in the Earth's temperature change since the last
glacial maximum, about 20,000 years ago. \pause
\begin{itemize}
\item  The first study to estimate the
temperature change was published in 1980
\item  Estimated a change of -1.5
degrees C, $\pm 1.2$ degrees C in tropical sea surface temperatures.
\item The
negative value means that the Earth was colder then than now.
\item Since 1980
there have been many other studies, which use different
measurement techniques, or proxies.
\item Some proxies can be used over land,
others over water.
\end{itemize}
\end{frame}

\begin{frame}
  \frametitle{Proxies}
   The 8 proxies used are

   \begin{enumerate}
   \item    "Mg/Ca"           1
   \item    "alkenone"        2
   \item    "Faunal"          3
   \item    "Sr/Ca"           4
   \item    "del 180"         5
   \item    "Ice Core"        6
   \item    "Pollen"          7
   \item    "Noble Gas"       8
   \end{enumerate}

\end{frame}

\begin{frame}[fragile]
  \frametitle{Variables}
  \begin{tiny}
\begin{verbatim}
  climate =
  read.table("http://www.stat.duke.edu/courses/Fall10/sta290/datasets/climate.dat",

  header=T)
\end{verbatim}

\end{tiny}
\pause
Each of the 53 studies reported
\begin{itemize}
\item {\tt deltaT} an estimate of the temperature change \pause
\item {\tt sdev}  a standard deviation of that
estimate \pause
\item {\tt proxy}  the proxy used (coded 1 to 8),  \pause
\item {\tt T.M} whether it was a
terrestrial or marine study (T/M), which is coded as 0 for
Terrestrial, 1 for Marine,  \pause
\item {\tt latitude} at which data were collected
\end{itemize}
\end{frame}

\begin{frame}
  \frametitle{Questions of Interest}
  \begin{enumerate}
  \item Do estimates vary systematically by proxy? \pause
  \item Do terrestrial estimates differ systematically from marine
    estimates? \pause
\item Do estimates vary systematically by latitude? \pause
\item Can we combine the studies to get a better estimate of the
  overall temperature change?  \pause
\item Are temperatures changing?  \pause
  \end{enumerate}

Build a larger  model or series of models to address these questions?

$$\E[\Delta T ] = f(\text{ Proxy, latitude})$$
\end{frame}

\begin{frame}
  \frametitle{Model Building}
  \begin{block}{George E. P. Box}
    Essentially, all models are wrong, but some are useful.
{\it Empirical Model-Building and Response Surfaces} (1987), co-authored with Norman R. Draper, p. 424
  \end{block} \pause
  \begin{itemize}
  \item
``true'' model may be a complicated function of latitude, proxy, as
well as other (omitted) covariates \pause
\item Assume that for each proxy $p$, there is a nonlinear relationship
  between $\Delta T$ and latitude $l$ and omitted variables $o$; $f(p, l, o)$ \pause
\item Taylor's series expansion about some point $l_0$:
  \begin{eqnarray*}
 f(p, l, o) & = &f(p, l_0, o) + (l - l_0)f'(p, l_0, o) + (l - l_0)^2 \frac{f"(p,
  l_0, o)}{2}\\
&  &  + R(l, p, o) \pause \\
f(p, l) & \approx &\beta_{p0} + \beta_{1p} l + \beta_{2p} l^2
\end{eqnarray*} \pause
\item Ignore $o$ and  remainder term
\end{itemize}
\end{frame}

\begin{frame}
  \frametitle{A Picture is Worth a Thousand Words (sometimes)}
\centerline{\includegraphics[height=3.5in]{temp-lat}}
\end{frame}
\begin{frame}
  \frametitle{ \R\ Model Formula}
\centerline{{\tt DeltaT $\sim$ proxy*poly(latitude, 2) }}
\pause
\begin{itemize}
\item Expand out predictors as\\
\small{{\tt proxy + poly(latitude, 2) + proxy:poly(latitude,
  2)}} \pause
\item {\tt proxy}  is a factor; default coding is to create 8  indicators
  of each proxy and then drop the column  associated with the first
  level of the factor  (MG/Ca in the example) \pause
\item {\tt poly(latitude, 2) } creates an orthonormal basis for a
  second order polymomial in latitude $[1, l, l^2]$ \pause
\item {\tt proxy:poly(latitude,  2)} takes the product of each of the 7 dummy
  variables for {\tt proxy} times the linear and quadratric terms for {\tt latitude}
\item Look at {\tt model.matrix( $\sim$ poly(latitude,2)*proxy, data=climate)}
\end{itemize}
\end{frame}

\begin{frame}[fragile]
  \frametitle{Estimates}
  \begin{tiny}
\begin{verbatim}
> summary(climate.lm)
Coefficients: (2 not defined because of singularities)
                               Estimate Std. Error t value Pr(>|t|)
(Intercept)                   -2.7933     2.3189  -1.205    0.235
Alkenone                       0.4463     2.3234   0.192    0.849
Faunal                         0.7235     2.4525   0.295    0.769
Sr/Ca                         -2.9254     2.5318  -1.155    0.255
Del180                        -0.3037     2.4030  -0.126    0.900
IceCore                       -3.1407     2.8504  -1.102    0.277
Pollen                        -2.6751     2.4528  -1.091    0.282
Noble Gas                     -3.2520     2.5698  -1.265    0.213
poly(latitude, 2)1            -3.0092    10.5916  -0.284    0.778
poly(latitude, 2)2            -7.3654    26.6516  -0.276    0.784
Alkenone:poly(latitude, 2)1    3.5493    10.6675   0.333    0.741
Faunal:poly(latitude, 2)1      6.5637    11.7978   0.556    0.581
Sr/Ca:poly(latitude, 2)1      11.8701    15.6097   0.760    0.451
Del180:poly(latitude, 2)1      0.8912    11.7526   0.076    0.940
IceCore:poly(latitude, 2)1         NA         NA      NA       NA
Pollen:poly(latitude, 2)1     -4.0769    13.5600  -0.301    0.765
Noble Gas:poly(latitude, 2)1  -8.7078    17.9962  -0.484    0.631
Alkenone:poly(latitude, 2)2    3.0832    26.6984   0.115    0.909
Faunal:poly(latitude, 2)2      2.8690    27.4056   0.105    0.917
Sr/Ca:poly(latitude, 2)2      19.2753    31.4567   0.613    0.543
Del180:poly(latitude, 2)2     16.1802    26.9623   0.600    0.552
IceCore:poly(latitude, 2)2         NA         NA      NA       NA
Pollen:poly(latitude, 2)2      3.3119    27.6753   0.120    0.905
Noble Gas:poly(latitude, 2)2  18.6612    30.0579   0.621    0.538

Residual standard error: 2.112 on 41 degrees of freedom
Multiple R-squared: 0.682,	Adjusted R-squared: 0.5191
F-statistic: 4.187 on 21 and 41 DF,  p-value: 4.382e-05

\end{verbatim}
  \end{tiny}
\end{frame}

\begin{frame}
  \frametitle{Conditional Plot}
{\tt coplot(deltaT $\sim$ latitude $\mid$ proxy, data=climate)}
  \centerline{\includegraphics[height=3in]{temp-lat-proaxy}}
\end{frame}

\begin{frame}
  \frametitle{Estimates and t-statistics}
  \begin{itemize}
  \item MLEs do not depend on the order of the variables in the model \pause
  \item regression coefficients are adjusted for the other variables
    in the model \pause
\item t-statistics  $$\frac{\lambda^T\b - \lambda^Tb_0}
{ \hat{\sigma} \sqrt{\lambda^T (\X^T\X)^{-1} \lambda}} \sim t(n - p, 0,
1)$$
under the hypothis $\lambda^Tb_0 = 0$
  \item t-values  correspond to test statistic for testing hypothesis
    H$_o$: $\beta_j = 0$ versus H$_a$: $\beta_j \neq 0$ given  the
    other variables are in the model \pause
  \item all p-values greater than $\alpha$ does not mean that all
    coefficients are zero! \pause
 \item redundancy \pause
 \item with factors use ANOVA for simultaneous testing \pause
  \end{itemize}
\end{frame}




\begin{frame}
  \frametitle{Decomposition}
  Consider a series of nested models: \pause
  \begin{eqnarray*}
    \M_0: \Y & = & \one_n \beta_0 + \eps   \pause\\
\M_1: \Y & = & \one_n \beta_0 + \X_1 \b_1 + \eps  \pause \\
\M_2: \Y & = & \one_n \beta_0 + \X_1 \b_1 + \X_2 \b_2 + \eps   \pause\\
\vdots & & \vdots \\
\M_k: \Y & = & \one_n \beta_0 + \X_1 \b_1 + \X_2 \b_2 + \ldots \X_k \b_k + \eps
  \end{eqnarray*} \pause

Let $\P_j$ denote the projection on the column space in each of the
models $\M_j$: $C(\X_0, \X_1, \ldots, \X_j)$ \pause
\begin{small}
  \begin{align*}
\| \Y \|^2 =&  \| \P_0 \Y\|^2 + \| (\P_1 - \P_0) \Y \|^2 + \|(\P_2
- \P_1)\Y\|^2 + \ldots \| (\P_k - \P_{k-1})\Y \|^2 + \\
& \|(\I_n -
\P_k)\Y\|^2
 \end{align*}
\end{small}

\end{frame}

\begin{frame} \frametitle{Likelihood Ratio Tests}

\begin{itemize}
  \item Compare likelihoods under $\M_{k}$ to $\M_{k-1}$ where $\M_{k-1} \subset \M_{k}$.
  $$\Lambda = \frac { \sup_{\theta_{k} \in \Theta_{k}} \cL(\theta_{k})}  { \sup_{\theta_{k-1} \in \Theta_{k-1}} \cL(\theta_{k-1})}$$
where  $\theta_j = (\beta_0, \ldots, \beta_j \sigma^2)$
\item  $\Lambda \ge 1$
\item If $\Lambda$ is "significantly" bigger than one, then should prefer the larger model (reject $\M_{k-1}$)
\item Pick a constant such that if  $\M_{k-1}$ is true, then $P(\Lambda > c \mid \M_{k-1}) = \alpha$, say $0.05$
\item Need distribution of $\Lambda$ (or transformation of it)

\end{itemize}



\end{frame}

\begin{frame} \frametitle{Likelihood Ratio Tests}

$$\Lambda = \frac { \sup_{\theta_{k} \in \Theta_{k}} \cL(\theta_{k})}  { \sup_{\theta_{k-1} \in \Theta_{k-1}} \cL(\theta_{k-1})}$$
\vspace{3in}

\end{frame}

\begin{frame}
  \frametitle{F tests}
The F statistic
$$F =  \frac{\| (\P_k - \P_{k-1})\Y \|^2 /(r(\P_{k}) - r(\P_{k-1})) }
{\hat{\sigma}^2} \sim F( r(\P_{k}) - r(\P_{k-1}),  n - p)$$
under the null hypothesis.\pause
\begin{itemize}
\item Numerator is a $\chi^2$ over df \pause
\item Denominator is a $\chi^2$ over df \pause
\item numerator and denominator are independent \pause

\item Nested models $C(M_k)$ contains $C(M_{k-1})$ \pause
\end{itemize}
\end{frame}

\begin{frame}
  \frametitle{Sequential F tests}

  \begin{tabular}{lllc} \hline
    Hypothesis* & SS & df & F \\ \hline
$\b_1 = 0$ & $\| (\P_1 - \P_0) \Y \|^2$ & $r(\P_1) - r(\P_0)$ & $
\frac{\frac{\| (\P_1 - \P_0) \Y \|^2} {r(\P_1) - r(\P_0) }}{\shat}$
\pause \\
$\b_2 = 0$ & $\| (\P_2 - \P_1) \Y \|^2$ & $r(\P_2) - r(\P_1)$ & $ \frac{\frac{\| (\P_
2 - \P_1) \Y \|^2} {r(\P_2) - r(\P_1) }}{\shat}$  \pause \\
$\vdots$ &$\vdots$ & $\vdots$& $\vdots$ \\
 $\b_k = 0$ & $\| (\P_k - \P_{k-1}) \Y \|^2$ & $r(\P_k) - r(\P_{k-1})$ & $ \frac{\frac{\| (\P_
k - \P_{k-1}) \Y \|^2} {r(\P_k) - r(\P_{k-1}) }}{\shat}$  \pause
  \end{tabular}
  \begin{itemize}
  \item Sequential test $\b_j = 0$ includes variables from the
    previous model $\b_0, \b_1,\ldots, \b_{j-1}$ but $\b_i$ for $i >
    j$ are all set to $0$  \pause
  \item All use estimate of $\shat =  \|(\I_n -
    \P_k)\Y\|^2 /(n - r(\P_k))$ under largest model
 \pause
  \item Unless $\P_j\P_i = \zero$ for $i \neq j$, decomposition will
    depend on the order of $\X_j$ in the model  \pause
  \item If last $\X_k$ is $n\times 1$, then $t^2 = F$  for testing H$_0$:
    $\beta_k = 0$  \pause
  \end{itemize}
\end{frame}




\begin{frame}[fragile]
  \frametitle{Order 1: Sequential Sum of Squares }
  \begin{small}
\begin{verbatim}
climate.lm = lm(deltaT ~ proxy *(poly(latitude,2)),
                weights=(1/sdev^2),
                data=climate)
anova(climate.lm)
Response: deltaT
                        Df  Sum Sq Mean Sq F value    Pr(>F)
proxy                    7 307.598  43.943  9.8541 3.848e-07 ***
poly(latitude, 2)        2  10.457   5.228  1.1725    0.3198
proxy:poly(latitude, 2) 12  74.065   6.172  1.3841    0.2126
Residuals               41 182.833   4.459
\end{verbatim}

\end{small}

\end{frame}

\begin{frame}[fragile]
 \frametitle{Order 2: Sequential Sum of Squares}
   \begin{small}
\begin{verbatim}
>anova(lm(deltaT ~ (poly(latitude,2))* proxy, weights=1/sdev^2,
          data=climate))
 Analysis of Variance Table

Response: deltaT
                        Df  Sum Sq Mean Sq F value    Pr(>F)
poly(latitude, 2)        2  79.869  39.935  8.9553 0.0005931 ***
proxy                    7 238.185  34.026  7.6304  6.93e-06 ***
poly(latitude, 2):proxy 12  74.065   6.172  1.3841 0.2125512
Residuals               41 182.833   4.459
\end{verbatim}
\end{small}
\end{frame}

\begin{frame}
  \frametitle{Prediction with Latitude}
  \centerline{\includegraphics[height=3in]{pred-temp-lat}}
\end{frame}

\begin{frame}[fragile]
  \frametitle{Multiple Model Objects and Anova in \R}
\begin{verbatim}
> anova(climate3.lm,climate2.lm,climate1.lm, climate.lm)
Analysis of Variance Table

Model 1: deltaT ~ T.M
Model 2: deltaT ~ poly(latitude, 2) + T.M
Model 3: deltaT ~ poly(latitude, 2) + proxy
Model 4: deltaT ~ proxy * (poly(latitude, 2))
  Res.Df    RSS Df Sum of Sq      F   Pr(>F)
1     61 385.66
2     59 347.11  2    38.542 4.3215 0.019814 *
3     53 256.90  6    90.215 3.3718 0.008552 **
4     41 182.83 12    74.065 1.3841 0.212551
---
Signif. codes:  0 '***' 0.001 '**' 0.01 '*' 0.05 '.' 0.1 ' ' 1
\end{verbatim}
\end{frame}


\begin{frame}[fragile]
  \frametitle{Other order}
 \begin{small}
\begin{verbatim}
> anova(climate3.lm,climate2.lm,climate1.lm, climate.lm)
Analysis of Variance Table

Model 1: deltaT ~ T.M
Model 2: deltaT ~ proxy
Model 3: deltaT ~ poly(latitude, 2) + proxy
Model 4: deltaT ~ proxy * (poly(latitude, 2))
  Res.Df    RSS Df Sum of Sq      F   Pr(>F)
1     61 385.66
2     55 267.35  6   118.301 4.4215 0.001555 **
3     53 256.90  2    10.457 1.1725 0.319767
4     41 182.83 12    74.065 1.3841 0.212551
---
Signif. codes:  0 '***' 0.001 '**' 0.01 '*' 0.05 '.' 0.1 ' ' 1
\end{verbatim}
  \end{small}
\end{frame}


\begin{frame}[fragile]
  \frametitle{Terrestrial versus Marine}
\begin{small}
\begin{verbatim}
climate.final = lm(deltaT ~ T.M + proxy -1, weights=(1/sdev^2))

               Estimate Std. Error t value Pr(>|t|)
T.MT            -5.6360     0.7132  -7.902 1.26e-10 ***
T.MM            -2.1145     0.4124  -5.127 3.93e-06 ***
proxyAlkenone   -0.1408     0.4381  -0.321    0.749
proxyFaunal     -0.1507     0.8971  -0.168    0.867
proxySr/Ca      -3.2188     0.7584  -4.244 8.49e-05 ***
proxyDel180     -0.6378     0.5048  -1.263    0.212
proxyIceCore     0.1360     1.3130   0.104    0.918
proxyPollen      0.5283     1.0033   0.527    0.601
proxyNoble Gas       NA         NA      NA       NA

Multiple R-squared: 0.9115,	Adjusted R-squared: 0.8986

          Df  Sum Sq Mean Sq  F value  Pr(>F)
T.M        2 2635.27 1317.63 271.0625 < 2e-16 ***
proxy      6  118.30   19.72   4.0561 0.00195 **
Residuals 55  267.35    4.86
\end{verbatim}
\end{small}
\end{frame}

\begin{frame}[fragile]
  \frametitle{Even Simpler ?}
  \begin{small}
\begin{verbatim}
lm(formula = deltaT ~ T.M + I(proxy == "Sr/Ca"), weights = (1/sdev^2))

                        Estimate Std. Error t value Pr(>|t|)
(Intercept)              -5.3915     0.4486 -12.018  < 2e-16 ***
T.MM                      3.0585     0.4649   6.579 1.30e-08 ***
I(proxy == "Sr/Ca")TRUE  -3.0003     0.6371  -4.709 1.52e-05 ***

Residual standard error: 2.166 on 60 degrees of freedom
Multiple R-squared: 0.5103,	Adjusted R-squared: 0.4939

Model 1: deltaT ~ T.M + I(proxy == "Sr/Ca")
Model 2: deltaT ~ T.M + proxy - 1
  Res.Df    RSS Df Sum of Sq      F Pr(>F)
1     60 281.58
2     55 267.36  5    14.228 0.5854  0.711
\end{verbatim}
\end{small}
\end{frame}

\begin{frame}
  \frametitle{Design}
  \centerline{\includegraphics[height=3in]{lat-proxy}}
\end{frame}

\begin{frame}
  \frametitle{Summary}
  \begin{itemize}
   \item Ignoring proxies, there are systematic trends with
     latitude. \pause
  \item Difference among proxies, even after adjusting for latitude \pause
\item Weak evidence of a latitude effect, after taking into account
  proxies (potential confounding)
\pause
 \item  Terrestrial sites differ from Marine sites, however there are significant difference among
      proxies within the Marine group  driven by the {\tt Sr/Ca} proxy which
      indicates a significantly greater increases in temperatures
\pause
   \item Significant warming for Terrestrial ($5.4 ^\circ C$) with
     Marine  sites   significantly  cooler ($3^ \circ  C$)\pause
   \item {\tt Sr/Ca} proxies are significantly cooler than other
     marine proxies by about $3^\circ C$ \pause
  \end{itemize}

Uncertainty Measures?  \pause  Normal Assumptions?
\end{frame}

\begin{frame}
  \frametitle{Added Variable Plots}
  \begin{enumerate}
  \item   Let $\P_{(-j)}$ denote the projection on the space spanned by
   $C(\X_0, \ldots, \X_{j-1}, \X_{j+1}, \ldots \X_k)$  (omit variable
    $j$) \pause
\item  Find residuals $\e_{\Y \mid \X_{(-j)}} = (\I - \P_{(-j)})\Y$
  from regressing $\Y$ on  all variables except $\X_j$ \pause
\item  Remove the effect of other explanatory variables from $\X_j$ by
  taking residuals $ \e_{\X_j \mid \X_{(-j)}} = (\I - \P_{(-j)})\X_j$ \pause
\item Plot $\e_{\Y \mid \X_{(-j)}}$ versus $\e_{\X_j \mid \X_{(-j)}}$ \pause
\item Slope is adjusted regression coefficient in full model $\mub \in
  C(\X_0, \ldots, \X_{j-1}, \X_j, \X_{j+1}, \ldots \X_k)$ \pause
\item {\tt library(car)} \pause
\item {\tt avPlots(climate1.lm, terms$=\sim .$)}
  \end{enumerate}
 \end{frame}

\end{document}
